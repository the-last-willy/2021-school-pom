\documentclass[a4paper]{article}

\usepackage{INTERSPEECH2021}

\usepackage{cases}
\usepackage{listings}
\usepackage[framemethod=tikz]{mdframed}

\hyphenpenalty=10000 % Disables hyphenation.

\title{Procedural terrain generation}
\name{Willy Jacquet$^1$, Parra Yoan$^2$}
%The maximum number of authors in the author list is twenty. If the number of contributing authors is more than twenty, they should be listed in a footnote or in acknowledgement section, as appropriate.
\address{
  $^1$Author Affiliation\\
  $^2$Co-author Affiliation}
\email{author@university.edu, coauthor@company.com}

\begin{document}

\maketitle
% 
\begin{abstract}
%Thus we propose a model base on mathematical logic and it's resulting C++ implementation. The model is comprehensive by extensibility and therefore faster development of procedural terrain by inferring basic function properties and automatically optimizing code.

\textit{Symbolic computation} is seen as a trade-off between expressiveness and performance.
Metaprogramming can minimize that overhead and enable further optimization. 


\end{abstract}

%\noindent\textbf{Index Terms}: speech recognition, human-computer interaction, computational paralinguistics

\section{Introduction}

In this paper we use procedural terrain generation as our primary application and thus tweaked the model for vector manipulation.

\section{State of the art}

\section{Symbolic computation}

Manipulation of \textit{functions} instead of \textit{values}.
\textit{Functions} are \textit{first class citizens}.

The following model is designed to take advantage of \textit{template metaprogramming} facilities. 

\begin{mdframed}
This document is thus extended with insight about making a \textit{C++} implementation out of it.
\end{mdframed}

\subsection{Generic functions}

A \textbf{reduction} predicate is introduced, noted $a \rightarrow b$ ($a$ can be \textit{reduced} to $b$). This predicate is \textit{transitive}.

\begin{mdframed}
A \textit{function} can be represented as a \textit{type} that holds no \textit{non-static data member}.
In particular, if two instances of functions share the same type then they refer to the same function.
\end{mdframed}

A \textbf{tuple} is defined as a \textit{sequence of objects} $(x_1,...,x_n)$.

The \textbf{application} of a function $f$ with $x_1,...,x_n$ is noted $f(x_1,...,x_n)$.
Generally, the \textit{application of a tuple} $(f_1,...,f_n)$ is defined as:
\begin{equation}
(f_1,...,f_n)(x_1,...,x_n) \rightarrow (f_1(x_1,...,x_n),...,f_n(x_1,...,x_n))
\end{equation}
If $f(x)$ cannot be further \textit{reduced} ($\neg\exists a, f(x) \rightarrow a$) then for $y_1,...y_n$:
\begin{equation}
f(x)(y_1,...y_n) \rightarrow f(x(y_1,...y_n))
\end{equation}

A \textbf{constant} $c$ is defined as a function that returns itself:
\begin{equation}
	c(x_1,...,x_n) \rightarrow c
\end{equation}

\begin{mdframed}
A \textit{compile-time constant} can be represented as:
\begin{lstlisting}
template<typename Type, Type Value>
struct constant {};
\end{lstlisting}
Special constants ($0$, $1$, $\pi$, $e$, ...) can be introduced as specific symbols since they appear in many (number) sets.
\begin{lstlisting}
struct zero {}; struct one {}; ...
\end{lstlisting}
\end{mdframed}

Given $i \in \mathbb{N}^*$, the \textbf{$i$-th projection} is a function that returns its $i$-th parameter:
\begin{numcases}{proj_i(x_1,...,x_n) \rightarrow}
	x_i          & \text{if $i \leq n$} \\
	proj_{i - n} & \text{otherwise} \label{partial_projection}
\end{numcases}
When the $i$-th projection is applied to the $i$-th parameter of each function, the notation is abbreviated:
\begin{equation}
	f(p_0,p_1,...) = f
\end{equation}

The \textbf{arity} (function) can be defined as follows:
\begin{equation}
\begin{split}
	arity(c) &= 0 \\
	arity(proj_i) &= i \\
	arity(f(g_1,...,g_n)) &= max\{arity(g_i)\}_{i \leq n}
\end{split}
\end{equation}

\subsection{Arithmetic functions}

\textit{Elementary arithmetic} functions are introduced with their usual definition:
\begin{equation}
\begin{split}
	(f + g)(x)      &\rightarrow f(x) + g(x) \\
	(f - g)(x)      &\rightarrow f(x) - g(x) \\
	(f \times g)(x) &\rightarrow f(x) \times g(x) \\
	(f / g)(x)      &\rightarrow f(x) / g(x)
\end{split}
\end{equation}
\textit{Partial application} is made possible by (\ref{partial_projection}). For example:
\begin{equation}
\begin{split}
	(proj_0 + proj_1)(x) &\rightarrow proj_0(x) + proj_1(x) \\
	                     &\rightarrow x + proj_0
\end{split}
\end{equation}
\textit{Lazy evaluation} is introduced by establishing:
\begin{equation}
\begin{split}
	0 \times g &\rightarrow 0\\
	f \times 0 &\rightarrow 0\\
	0 / g      &\rightarrow 0
\end{split}
\end{equation}
\begin{mdframed}
This can be implemented by providing \textit{function overloads} having the first or second operand with the zero type.
\end{mdframed}

The current model can be extended with any usual function.
Any function that is not mentioned in this section is defined with its usual meaning.

\subsection{Differential calculus}

The \textbf{partial derivative} \textit{with respect to $x$} is introduced for generic functions:
\begin{align}
\frac{\partial}{\partial x}(c) &\rightarrow 0 \text{ with $c$ a constant} \\
\frac{\partial}{\partial proj_j}(proj_i) &\rightarrow
\begin{cases}{}
1 & \text{if $i = j$} \\
0 & \text{otherwise}
\end{cases} \\
\frac{\partial}{\partial x}(f(y)) &\rightarrow \frac{\partial}{\partial x}(y) \times \frac{\partial}{\partial x}(f)(y)
\end{align}
For arithmetic functions:
\begin{align}
\frac{\partial}{\partial x}(f + g) &\rightarrow \frac{\partial}{\partial x}(f) + \frac{\partial}{\partial x}(g) \\
\frac{\partial}{\partial x}(f - g) &\rightarrow \frac{\partial}{\partial x}(f) - \frac{\partial}{\partial x}(g) \\
\frac{\partial}{\partial x}(f \times g) &\rightarrow \frac{\partial}{\partial x}(f) \times g + f \times \frac{\partial}{\partial x}(g) \\
\frac{\partial}{\partial x}(f / g) &\rightarrow \frac{\frac{\partial}{\partial x}(f) \times g - f \times \frac{\partial}{\partial x}(g)}{g \times g}
\end{align}

\subsection{Computational redundancy}

Suppose that, provided a function $f \times g$, you wish to compute its value \emph{and} partial derivative for an argument $x$.

\subsection{Parallelization}

\subsection{Compilation}

\subsection{Performance}

\section{Common functions}

\subsection{Interpolation}

Provided an interpolant $t \in [0, 1]$.
Polynomial interpolation with null derivatives is defined as follows:
\begin{itemize}
\item linea:r $t$
\item cubic: $-2t^3 + 3t^2$
\item quintic: $6t^5 - 15t^4 + 10t^3$
\end{itemize}

\subsection{Noise functions}

\textit{Coherent noise} can be used to model many natural phenomena and is characterized as follow \cite{libnoise_coherent_noise}:
\begin{itemize}
\item \textit{Referential transparency} i.e.\ the same input produces the same output
\item A small change in the input value will produce a small change in the output value.
\item A large change in the input value will produce a random change in the output value
\end{itemize}

\textbf{Value noise} is achieved by interpolating random values on a regular grid along each axis until it is reduced to a single number.

\begin{figure}[h]
\includegraphics[width=\linewidth]{img/value_noise_2d}
\caption{2D value noise with quintic interpolation.}
\end{figure}


\begin{lstlisting}[caption=Pseudo code implementation of 2D value noise]
float noise(v:float[2]):
  i=floor(v)
  t=fract(v)
  return interp(
    interp(hash(i+[0,0]),hash(i+[0,1]),t[0]),
    interp(hash(i+[1,0]),hash(i+[1,1]),t[0]),
    t[1])
\end{lstlisting}
With $interp$ being an interpolation function. 

\subsection{Worley noise}

\section{Terrain representation}

A \textbf{terrain} can be defined as a $C^2$ function $\mathbb{R}^2 \rightarrow \mathbb{R}$.
Therefore it is interpreted 

\subsection{Coloration}

\subsection{Rendering}

\subsection{Implicit representation}



\section{Results}

\section{Appendix A: Derived functions}

Many usual functions can be defined in terms of those previously introduced.
Doing so makes inferring most properties (derivative, domain, ...) automatically possible.

\subsection{Generic functions}
\begin{flalign*}
swap = (proj_1, proj_0)
\end{flalign*}

\subsection{Arithmetic functions}
\begin{equation*}
\begin{array}{ll}
opposite    &= 0 - proj_0 \\
inverse     &= 1 / proj_0 \\
\\
translation &= p1(p0 + p2) \\
scaling     &= p1(p0 \times p2)
\end{array}
\end{equation*}

\subsection{Calculus functions}
\begin{equation*}
\begin{array}{ll}
\partial(cos) &= - sin \\
\partial(sin) &= cos \\
\end{array}
\end{equation*}

\bibliographystyle{IEEEtran}
\bibliography{mybib}

\end{document}
